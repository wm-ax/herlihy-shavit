\paragraph{9.} Define r-bounded waiting for a given mutual exclusion algorithm to
mean that if ${D^j}_A \to {D^k}_B$ then ${CS^j}_A \to {CS^{k+r}}_B$. 
Is there a way to define a doorway for the Peterson algorithm such that it provides $r$-bounded waiting for some value of $r$?

No: between one thread's write to \textsc{victim} on line 9 and its read of \textsc{victim} on line 10, there is no bound on the number of times another thread may execute its critical section.

\paragraph{10.} Suppose that in the definition of fairness, we replace the reference to ``doorway'' with a reference to the initial instruction $I$ of the lock method.  We thus get
\begin{itemize}
\item[] A lock is \emph{first-come-first-served-1} if, whenever thread $A$ finishes $I$ before thread $B$ starts $I$, then $A$ cannot be overtaken by $B$.  
\end{itemize}
I will argue that \emph{no} lock method can be first-come-first-served-1, because one thread can always ``overtake'' another.  

More precisely, let's write ${I_A}^k$ for the time interval of $A$'s $k$th execution of $I$, and similarly write ${I_B}^k$.  I will also write $X\hookrightarrow Y$ to mean that $X\to Y$, but also that no instruction were executed in between. 
Because $I_A$ and $I_B$ are discrete operations, it should be possible that ${I_A}^j\hookrightarrow {I_B}^k$.
Suppose that it is possible that ${I_A}^j\hookrightarrow {I_B}^k$ and ${CS_A}^j\to {CS_B}^k$ (and similarly possible with $A, B$ everywhere interchanged). I will argue that it must then also be possible that ${I_A}^j\to {I_B}^k$ while ${CS_B}^k\to {CS_A}^j$.

To this end, I will assume the following: (*) if the state of the machine following ${I_A}^j\hookrightarrow {I_B}^k$ is the same as the state of the machine following ${I_B}^k\hookrightarrow {I_A}^j$, then it is possible for either thread to overtake the other.  

If $I$ is a read, then it is clear that the machine state following ${I_A}^j\hookrightarrow {I_B}^k$ is the same as the machine state following ${I_B}^k\hookrightarrow {I_A}^j$.  So by (*), $B$ can overtake $A$.

On the other hand, suppose that $I$ is a write.  Let's use $L_A(I)$ to denote the location addressed by thread $A$ in executing $I$, and likewise for $L_B(I)$.  Now the problem splits into two subcases.

If $L_A(I)=L_B(I)$.  Then after $A$'s execution of $I$, thread $B$ executes $I$ as well, and also writes something to $A$.  In this way, $B$ obliterates whatever trace $A$ has left in $L_A(I)$ of having begun the lock method.  So again, $B$ cannot tell whether $A$ has entered the lock method.

Finally suppose that $L_A(I)\neq L_B(I)$.  Then, the machine state following ${I_A}^j\hookrightarrow {I_B}^k$ is the same as the machine state following ${I_B}^k\hookrightarrow {I_A}^j$, so the conclusion again follows by (*).

\paragraph{11.} Consider the algorithm

\begin{verbatim}
1 class Flaky implements Lock {
2   private int turn;
3   private boolean busy = false;
4   public void lock() {
5     int me = ThreadID.get();
6     do {
7       do {
8         turn = me;
9       } while (busy);
10      busy = true;
11    } while (turn != me);
12   }
13 public void unlock() {
14   busy = false;
15   }
16 }
\end{verbatim}

a. Flaky does satisfy mutual exclusion.  For, suppose that $A, B$ are in the critical section simultaneously.  Without loss of generality, suppose that $A$ was the latest to read from $\textsc{Turn}$.  Since $A$ then read from this a different value than $B$ last did, $A$ must have last written to $\textsc{Turn}$ after $B$ read from it.  But then, $B$ must have last written $\textsc{Busy}$ to be true \emph{before} $A$ read busy to be false, but between $A$'s last read of $\textsc{Busy}$ and $B$'s last write to it, neither thread otherwise writes to it.  Hence, $A$ read the value true to be false, a contradiction. 

b. It is not starvation free.  Immediately after $A$ writes $\textsc{Turn}$ to be $A$'s, it is possible for $B$ to run through the lock method, the critical section, and the unlock method unboundedly many times; and $A$'s progress will be blocked provided it is unlucky enough each time to try to read from $\textsc{Busy}$ immedately after $B$ has written it to be true.

c. So, suppose that after $A$ writes $\textsc{Busy}$ to be true but before $A$ reads from $\textsc{Turn}$,  $B$ writes $\textsc{Turn}$ to be $B$'s.  Then both of $A$ and $B$ will enter their inner loops while $\textsc{Busy}$ has been set to true, and neither will make further progress.

12. The Filter lock allows some threads to overtake others an arbitrary number of times.  Suppose, anyway, that there are three threads $A, B, C$.  Let $A$ first pass through its doorway at level 1.  At this point and before $A$ completes a spin cycle, the following can happen arbitrarily many times: thread $B$ then passes through the level 1 doorway, thread $C$ passes through the doorway; thread $B$ advances through to the critical section and unlocks; thread $C$ advances through to the critical section and unlocks.

13. I will say that a thread $A$ \emph{passes through} a lock $L$ when $A$ has completed the $L$'s lock method but $A$ has not released $L$ (so $A$ has still flagged $L$).  I will argue that if two threads pass through $L$, then two threads also pass through some child of $L$.  To see this, note that by the mutual exclusion property of the Peterson lock, some third thread must have called the lock method of $L$.  Now, $L$ cannot be a leaf lock, because only two threads are assigned to a leaf.  Therefore, $L$ must have two children, and two of the three threads must have passed through one of them.  

It follows that if two threads pass through a single lock, then the tree lock contains an infinite descending chain of locks, which is absurd.

Therefore: no two threads pass through a single lock.  In particular, no two threads pass through the root lock.  However, a thread in the critical section passes through the root lock.  So, two threads cannot both be in the critical section.

